\documentclass[a4paper,10pt]{scrartcl}
\usepackage[utf8]{inputenc}

\usepackage{amsmath}
\usepackage{booktabs}
\usepackage{graphicx}
\usepackage[margin=15mm]{geometry}

\footskip=20pt


%opening
\title{Validation of Bellerophon for the cavity test case}
\author{Andreas Groß}

\begin{document}

\maketitle

\section{Case Description}
This test case compares the results for a steady laminar flow in a lid-driven
cavity in two dimensions with \(Re=10^2\). Dimensions of the cavity are 0.1 m
by 0.1 m and the inlet velocity of the lid is 1 \(\frac{\text m}{\text s}\).

The single grid case is discreticised with 30 cells in each direction. For the
overset case, the mesh is divided in the horizontal plane and discreticised with
30 cells in the horizontal direction and 18 cells in the vertical direction with
an overlap of 2.5 cells.

\section{Results}

The velocity distribution for the overset and single grid are shown in
Figure~\ref{U}.
\begin{figure}[!htbp]
  \centering
  \resizebox{0.49\columnwidth}{!}{\input{../results/overset_U.tex}}
  \resizebox{0.49\columnwidth}{!}{\input{../results/single_U.tex}}
  \caption{Magnitude of velociy}
  \label{U}
\end{figure}

A comparison of isobars and isotachs for the calculations is shown in
Figure~\ref{iso}.
\begin{figure}[!htbp]
  \centering
  \resizebox{0.49\columnwidth}{!}{\input{../results/isobars.tex}}
  \resizebox{0.49\columnwidth}{!}{\input{../results/isotachs.tex}}
  \caption{Isobars and isotachs}
  \label{iso}
\end{figure}

The convergence history is shown in Figure~\ref{residuals}.
\begin{figure}[!htbp]
  \centering
  \resizebox{0.49\columnwidth}{!}{\input{../results/residuals.tex}}
  \caption{Covergence history}
  \label{residuals}
\end{figure}

A comparison of the computational time is given in Figure~\ref{t}.
\begin{figure}[!htbp]
  \centering
  \resizebox{0.49\columnwidth}{!}{\input{../results/times.tex}}
  \caption{Computational time}
  \label{t}
\end{figure}
\end{document}
